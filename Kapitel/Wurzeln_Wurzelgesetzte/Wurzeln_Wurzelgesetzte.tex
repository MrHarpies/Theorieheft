\section{Wurzeln und Wurzelgesetze}
Wurzeln sind ein wesentlicher Bestandteil in der Mathematik. Das Grundkonzept hinter Wurzeln ist, dass Umkehren einer Potenz, in der eine Zahl so oft mit sich Multipliziert wird, wie der Exponent angibt. Die $n$-te Wurzel aus einer Zahl $a$ ist genau die Zahl, die $n$-mal mit sich selbst multipliziert den Wert $a$ ergibt.
Man Schreibt: $\sqrt[n]{a}$, wobei $n$ angibt, wie oft die Zahl mit sich selbst multipliziert wurde, damit sich $a$ ergibt. 

\begin{beispiel}
	\begin{align}
		\sqrt[3]{8} \Rightarrow 2\cdot 2\cdot 2&=8\\
		\sqrt[5]{243} \Rightarrow 3\cdot 3\cdot 3\cdot 3\cdot 3&=243
	\end{align}
\end{beispiel}
\subsection{Wurzelgesetze} Auch mit Wurzeln kann gerechnet werden, deswegen finden gewisse Rechengesetze auch hier einen Anwendung.
\begin{enumerate}
	\item Wurzel ziehen aus einer Division. \[\sqrt{\frac{a}{b}}=\frac{\sqrt{a}}{\sqrt{b}}\]	
	\item Wurzel ziehen aus einer Multiplikation \[\sqrt{a\cdot b}=\sqrt{a}\cdot \sqrt{b}\]
\end{enumerate}