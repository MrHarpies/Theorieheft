\section{Gleichungen}\label{sec:Gleichungen}
 Eine Gleichung besteht immer aus zwei Termen, die mit einem Gleichsetzungszeichen verknüpft sind. Damit wird ausgedrückt, dass beide Seiten, der Gleichung, den gleichen Wert haben bzw. gleichschwer sind. Taucht eine oder mehrere Variablen auf einer oder beider Seiten, der Gleichung auf, so ist genau der Wert der Variable eine Lösung der Gleichung für den das Gleichheitszeichen stimmt. 
Eine Gleichung kann eine Lösung, mehrere Lösungen oder keine Lösung haben. 

\begin{beispiel}
	\begin{enumerate}
		\item Eine Lösung 
			\begin{align*}
				3x-5&=2x+3\\
				&=\{8\}
			\end{align*}
		\item mehrere Lösungen 
			\begin{align*}
				2a+b&=7\\
				&=\{(3.1);(2.3)...\}
			\end{align*}
		\item Keine Lösung
			\begin{align*}
				x^2&=-4\\
				&=\{\}
			\end{align*}
	\end{enumerate}
\end{beispiel}
\subsection{Äquivalenzumformung}\label{sec:Gleichungen/Aequivalenzumformung}
Eine Äquivalentzumformnung ist eine Umformung, die auf beiden Seiten einer Gleichung durchgeführt wird. Dabei wird die Kernaussage der Gleichung nicht verändert, die Darstellung allerdings schon.

\begin{beispiel}
	\begin{alignat*}
		3x-7&=9\qquad &| +7\\
		3x&=16 \qquad &|:3\\
		x&=\frac{16}{3}
	\end{alignat*}
\end{beispiel}
\subsection{Schema zum Lösen von Gleichungen}\label{sec:Gleichungen/Schema zum Loesen von Gleichungen}
Liegt eine Gleichung vor, an der der höchste Exponent $n:$$1$ ist, kann man zu der Ermittlung der Lösung wie folgt vorgehen . 
\begin{enumerate}
	\item Klammern auflösen und Zusammenfassen mit den Distributivgesetzen und der binomischen Formel.
	\item Alles mit einer Variable auf eine Seite bringen
	\item Alles ohne Variable auf die andere Seite bringen
	\item Normieren, indem man auf ein $x$ runter oder hochrechnet. 
\end{enumerate}
\setcounter{equation}{0}

\begin{beispiel}
	\begin{align}
		(x-2)&=(x+3)(x-3)\\
		x^2-4x+4x&=x^2-9\\
		-4x+4 &= -9\\
		-4x&=-13\\
		x&=\frac{13}{4}
	\end{align}
\end{beispiel}\\
Wobei in der ersten Zeile die Ausgangsgleichung steht. Anschließend folgen die Schritt, wie oben beschrieben. 