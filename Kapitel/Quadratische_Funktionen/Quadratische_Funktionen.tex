\section{Quadratische Funktionen}\label{sec:Quadratische Funktionen}
Eine quadratische Funktion ist eine weitere Art von Funktionen. Sie stellt allerdings nicht wie bei den linearen Funktionen einen linearen Sachverhalt dar, sondern einen quadratischen. 
\subsection{Normalformen}\label{sec:Quadratische Funktionen/Normalform}
\subsubsection{Scheitelpunktform}
Der Scheitelpunkt einer quadratischen Funktion oder auch einer Parabel ist der größte bzw. kleinste Wert einer parabolischen Funktion. Die allgemeine Form der Scheitelpunktform einer quadratischen Funktion ist die folgende. 
\begin{align*}
	f(x)=a(x-d)^2+e
\end{align*}
Dabei gilt
\begin{itemize}
	\item Der Streckfaktor $a$ bestimmt, ob die Parabel nach oben ($a>0$) oder nach unten ($a<0$) geöffnet ist. Auch bestimmt er, wie die Parabel gestreckt ($|a|>0$) oder gestaucht $(0<|a|<1)$ ist
	\item Mit dem Parameter $d$ wird angeben, wie weit die Funktion nach rechts bzw. links verschoben wird. Hierbei muss allerdings auf das Vorzeichen von $d$ geachtet werden. Ist $d>0$, so verschiebt sich der Graph nach links, während sich der Graph bei $d<0$ nach rechts verschiebt.
	\item Der Parameter $e$ beschreibt die Verschiebung auf der $Y$-Achse. 
\end{itemize}
Zu dem obig behandelten Thema steht eine digitale Visualisierung im Internet bereit. \href{https://www.geogebra.org/m/wfekxgxw}{\ExternalLink}

\subsubsection{Faktorisierte Form}\label{sec:Quadratische Funktionen/Faktorisierte Form}
\subsubsection{Normalform} \label{sec:Quadratische Funktionen/Normalform/Faktorisierte Form}

\subsection{Scheitelpunktform herstellen}\label{sec:Quadratische Funktionen/Scheitelpunktform herstellen}
Ist eine Funktion $f(x)$ in der Form $f(x)=ax^2+bx+c$, so kann diese in die Form der Scheitelpunktform gebracht werden. Hierfür wendet man die zweite binomische Formel rückwärts an. Dabei unterscheidet man in zwei Fällen.  
\begin{itemize}
	\item Ist $c=\left(\frac{b}{2}\right)^2$, so kann eine Funktion deren Form der Normalform entspricht in die Scheitelpunktform umgeschrieben werden.$x^2+bx+c=\left(x\frac{b}{2}\right)^2$
	\item Ist $c\neq\left(\frac{b}{2}\right)^2$, so
\end{itemize}
%ToDo Hier fehlt Inhalt bezüglich der quadratischen Ergänzung
\subsection{Funktion anhand des Graphen ablesen}\label{sec:Quadratische Funktionen/Funktion anhand des Graphen ablesen}
Um die Funktion eines parabolisch ausgeprägten Graphen abzulesen benötigt man ebenfalls die Scheitelpunktform, in die die jeweilgen Punkte des Graphen eingesetzt werden. Zunächst setzt man hierfür den Wert für $d$ und $e$ ein. Anschließend wird sich ein ablesbarer Punkt ausgesucht, der für restlichen Parameter eingesetzt wird. 

\begin{beispiel}
	Auf der Abbildung befindet sich der Scheitelpunkt bei den Koordianten $(5;10)$. Setzt man nun diesen Punkt in die Scheitelpunktform ein, so ist das Einsetzten eines Punktes der abschließende Schritt.
	\begin{align*}
	\setcounter{equation}{0}
		&S(5;10)\\
		&P(0;10)\\
		f(x)&=a(x-5)^2+10\\
		10&=a(0-5)^2+10\\
		10&=-25a+10\\
		10+25a&=10
	\end{align*}
\end{beispiel}
\subsection{Lösen einer quadratischen Gleichung}\label{sec:Quadratische Funktionen/Loesen einer quadratischen Gleichung}
Ist eine Gleichung mit einer Variablen deren höchster Exponent 2 in der Form $ax^2+b=0$ oder $ax^2+bx+c=0$ gegeben, so spricht man von einer reinen quadratischen bzw. von einer gemischten quadratischen Gleichung. Um die Lösung für solch eine Gleichung zu bestimmen, gibt es zwei unterschiedliche Vorgehensweisen.
\subsubsection{Vorgehen mit einer reinen quadratischen Gleichung}
\begin{enumerate}
	\item Grundform herstellen, indem die Gleichung gleich $0$ gesetzt wird.
	\item $x^2$ isolieren mithilfe von Umformungen
	\item $\pm\sqrt{}$
\end{enumerate}
\begin{beispiel}
	\begin{align*}
		(4x-1)^2&=(x-4)^2\\
		16x^2-8x+1&=x^2-8x+16\\
		15x^2-15&=0\\
		15x^2&=15\\
		x^2&=1\\
		&\Rightarrow x_1=1, x_2=-1
	\end{align*}
\end{beispiel}
\subsubsection{Vogehen mit einer gemischten quadratischen Gleichung}
\begin{enumerate}
	\item Wenn ein Faktor vor dem $x$ steht, muss dieser druch Division entfernt werden. 
	\item Anwendung von quadratischer Ergänzung, PQ-Formel oder Mitternachtsformel
\end{enumerate}
\paragraph{Quadratische Ergänzung} Bei der quadratischen Ergänzung wird der Teil, der bei der Normalform als $b$ bezeichnet wird, halbiert und anschließend quadriert. Diese Zahl wird zwischen $b$ und $c$ geschoben mit in der Form $b^2-b^2$. Anschließend ergibt sich mit den ersten drei Teilen der quadratischen Gleichung eine bekannte Form. Diese sollte der Form der einer binomischen Formel ergeben. Anschließend kann die jeweilige binomische Formel rückwärts angewendet werden. 

\begin{beispiel}
	\begin{align*}
		8x^2+2x-3&=0\tag{$:8$, sodass $x^2$ normiert wird}\\
		x^2+\frac{1}{4}x-\frac{3}{8}&=0\tag{$+\frac{3}{8}$ alle Summanden ohne $x$ nach rechts}\\
		x^2+\frac{1}{4}x&=\frac{3}{8}\tag{$\left(\frac{1}{8}\right)^2$ wird als $\frac{b}{2}$ ergänzt \& hinzufügen beide Seiten}\\
		x^2+\frac{1}{4}x+\left(\frac{1}{8}\right)^2&=\frac{2}{8}+\frac{1}{64}\tag{Anwendung der binomischen Formeln}\\
		\left(x+\frac{1}{8}\right)^2&=\frac{24}{64}+\frac{1}{64}\\
		\left(x+\frac{1}{8}\right)^2&=\frac{25}{64}\tag{$\sqrt{ }$ Plus Minus Wurzel ziehen }\\
		x_1= x+\frac{1}{8}&=\frac{5}{8}\tag{$-\frac{1}{8}$}\\
		x_2= x+\frac{1}{8}&=-\frac{5}{8}\\
		&\Rightarrow x_1=\frac{1}{2}, x_2=-\frac{3}{4}
	\end{align*}
\end{beispiel}
\paragraph{PQ-Formel} Bei der Anwendung der PQ-Formel, werden die Zahlen, die anstelle von den Variablen $b$ und $c$ in der Normalform stehen, als $p$ und $q$ definiert und in die folgende Form eingesetzt. 
\begin{align*}
	-\frac{p}{2} \pm\sqrt{\bigg(\frac{b}{2}\bigg)^2-q}
\end{align*}

\begin{beispiel}
	\begin{align*}
		3x^2-7x+2&=0\tag{:3 $x^2$ wird normiert}\\
		x^2-\frac{7}{3}x+\frac{2}{3}&=0\tag{$p=-\frac{7}{3},\ q=\frac{2}{3}$}\\
		-\left(-\frac{7}{3}\right) \pm\sqrt{\left(\frac{-\frac{7}{3}}{2}\right)^2-\frac{2}{3}}\tag{Einsetzen in PQ-Formel}\\
		x_1&= \frac{7}{6}+\frac{5}{6}\\
		x_2&= \frac{7}{6}-\frac{5}{6}\\
	\end{align*}
\end{beispiel}