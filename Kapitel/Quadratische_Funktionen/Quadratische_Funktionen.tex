\section{Quadratische Funktionen}
Eine quadratische Funktion ist eine weitere Art der Funktionen. Sie stellt allerdings nicht wie bei den linearen Funktionen einen linearen Sachverhalt dar, sondern einen quadratischen. 
\subsection{Normalformen}
\subsubsection{Scheitelpunktform}
Der Scheitelpunkt einer quadratischen Funktion oder auch einer Parabel ist der größte bzw. kleinste Wert einer parabolischen Funktion. Die allgemeine Form der Scheitelpunktform einer quadratischen Funktion ist die folgende. 
\begin{align*}
	f(x)=a(x-d)^2+e
\end{align*}
Dabei gilt
\begin{itemize}
	\item Der Streckfaktor $a$ bestimmt, ob die Parabel nach oben ($a>0$) oder nach unten ($a<0$) geöffnet ist. Auch bestimmt er, wie die Parabel gestreckt ($|a|>0$) oder gestaucht ($(0<|a|<1)$) ist
	\item Mit dem Parameter $d$ wird angeben, wie weit die Funktion nach rechts bzw. links verschoben wird. Hierbei muss allerdings auf das Vorzeichen von $d$ geachtet werden. Sei $d>0$ verschiebt sich der Graph nach links, während sich der Graph bei $d<0$ nach rechts verschiebt.
	\item Der Parameter $e$ beschreibt die Verschiebung auf der $Y$-Achse. 
\end{itemize}
Zu dem obig behandelten Thema steht eine digitale Visualisierung im Internet bereit. \href{https://www.geogebra.org/m/wfekxgxw}{\ExternalLink}

\subsubsection{Faktorisierte Form}
\subsubsection{Normalform}

\subsection{Scheitelpunktform herstellen}
Ist eine Funktion $f(x)$ in der Form $f(x)=x^2+b+c$, so kann diese in die Form der Scheitelpunktform gebracht werden. Hierfür wendet man die zweite binomische Formel rückwärts an. Dabei unterscheidet man in zwei Fällen.  
\begin{itemize}
	\item Ist $c=\big{(}\frac{b}{2}\big{)}}^2$, so kann eine Funktion deren Form der Normalform entspricht in die Scheitelpunktform umgeschrieben werden.\\ $x^2+bx+c=\big{(}x\frac{b}{2}\big{)}^2$
	\item Ist $c\neq\big{(}\frac{b}{2}\big{)}}^2$, so
\end{itemize}
% Hier fehlt Inhalt bezüglich der quadratischen Ergänzung
\subsection{Funktion anhand des Graphen ablesen}
Um die Funktion eines parabolisch ausgeprägten Graphen abzulesen benötigt man ebenfalls die Scheitelpunktform, in die die jeweilgen Punkte des Graphen eingesetzt werden. Zunächst setzt man hierfür den Wert für $d$ und $e$ ein. Anschließend wird sich ein ablesbarer Punkt ausgesucht, der für restlichen Parameter eingesetzt wird. 

\begin{beispiel}
	Auf der Abbildung befindet sich der Scheitelpunkt bei den Koordianten $(5;10)$. Setzt man nun diesen Punkt in die Scheitelpunktform ein, so ist das Einsetzten eines Punktes der abschließende Schritt.
	\begin{align}
	\setcounter{equation}{0}
		&S(5;10)\\
		&P(0;10)\\
		f(x)&=a(x-5)^2+10\\
		10&=a(0-5)^2+10\\
		10&=-25a+10\\
		10+25a&=10
	\end{align}
\end{beispiel}

