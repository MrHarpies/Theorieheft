
\section{Rechengesetze}\label{sec:Rechengesetzte}
Die Rechengesetze sind grundlegend für die Anwendung vieler Rechenverfahren. Sie beschreiben Grundlegende Regeln, die ausnahmslos gelten.

\subsection{Distributivgesetz}\label{sec:Distributivgesetzt}
Das Distributivgesetz ist auch unter dem Namen Verteilungsgesetz bekannt. Hierbei wird ein Faktor vor einer Klammer auf alle Inhalte einer Klammer verteilt. 

\begin{beispiel}
	\[a\cdot(b+c)=a\cdot b+a\cdot c\]
\end{beispiel}
\subsection{Erweitertes Distributivgesetz}\label{sec:Erweitertes Distributivgesetz}
Das erweiterte Distributivgesetz wird auf die Multiplikation von zwei Klammern miteinander, so kann man diese auflösen, indem man das einfache Distributivgesetz mehrfach hintereinander anwendet. 

\begin{beispiel}
	\begin{enumerate}
		\item Mit Variablen: \[(a+b)\cdot(c+d)=(a+b)\cdot c+(ac+b)\cdot d=ac+bc+ad+ad+bd\]
		\item Mit Zahlen: \[(3a-6c)\cdot (-2a+3b)=-6a^2+9ab+12ca-18cb\]
	\end{enumerate}
\end{beispiel}
\pagebreak
\subsection{Faktorisieren}\label{sec:Faktorisieren}
In manchen Fällen ist es sinnvoll, einen Term mit einem Produkt zu verwandeln. Diesen Vorgang nennt man Faktorisieren. Dabei sucht man seine Variable oder einen Teiler, der in jedem Summanden des Terms vorkommt und zieht diesen aus dem Term heraus, indem man das Distributivgesetz ''rückwärts'' anwendet.
Man schreibt: $\color {red}a\color {black}b+\color {red}a\color {black}c+\color {red}a\color {black}d=\color {red}a\color {black}\cdot b+\color {red}a\color {black}\cdot c+\color {red}a\color {black}\cdot d=\color {red}a\color {black}\cdot(b+c+d)$

\begin{beispiel}
	\begin{enumerate}
		\item 
			\begin{align*}
				&3\color{red}a\color{black}-6\color{red}a\color{black}b+4\color{red}a\color{black}c\\
				&=\color{red}a\color{black}\cdot3+\color{red}a\color{black}\cdot6b+\color{red}a\color{black}\cdot4c\\
				&=\color{red}a\color{black}\cdot(3-6b+4c)
			\end{align*}
		\item 
			\begin{align*}
				&6ac-8ab+10a^2\\
				&=2a\cdot 3c-2a\cdot 4b+2a\cdot5a\\
				&=(3c-4b+5a)\cdot2a
			\end{align*}
		\item 
			\begin{align*}
				&x-2x^2+3x^3\\
				&=x\cdot 1-x\cdot 2x+x\cdot 3x^2
			\end{align*}
	\end{enumerate}
\end{beispiel}
\pagebreak



