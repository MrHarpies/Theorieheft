\section{Lineare Funktionen}
Wird eine Funktion durch eine Gleichung in der Form $y=mx+b$ dargestellt, so spricht man von einer linearen Funktion. Dabei gibt $b$ den Schnittpunkt mit der $Y$-Achse, den sogenannten $Y$-Achsenabschnitt an. DIe Variable $m$ ist die Steigung der Funktion. Hat $m$ die Form $m=\frac{a}{b}$, so gibt $b$ den Weg auf der $X$-Achse und $a$ den Weg auf der $Y$-Achse an, um von einem Punkt zu dem nächsten zu gelangen. Dies kann gut mit dem Steigungsdreieck darsgestellt werden. Beim Einzeichnen ist zu beachten, dass immer erst nach Rechts und anschließend nach oben bei potiven Zahlen und nach unten bei negativen Zahlen gegangen wird.
\subsection{Parallelität} Um die Eigenschaft der Parallelität zu bestimmen bei einer linearen Funktion, betrachtet man den Steigungsfaktor $m$.
Im Bezug auf Parallelität ist folgendes festzuhalten. 
\begin{align*}
	f(x)=mx+b\\
	a(x)=m_1x+b
\end{align*}
Sind die Werte, die $m$ annimmt, die selben, so spricht man von einer Parallelität.
\subsection{Orthogonalität} Im Vergleich zu der Parallelität stellt die Orthogonalität das Gegenteil da. Eine Funktion, die zu einer anderen Orthonogal steht, bildet mit der anderen Funktion einen Schnittpunkt im 90\degree