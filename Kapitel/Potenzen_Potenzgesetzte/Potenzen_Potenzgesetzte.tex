\section{Potzen}\label{sec:Potenzen}
Eine Potzen besteht immer aus einer Basis und einem Exponenten. Dabei gibt der Exponenten an, wie oft die Basis mit sich selbst multipliziert wird. Man schreibt $a^b$. 
\subsection{Potenzgesetzte}\label{sec:Potenzen/Potenzgesetze}
 Auch mit Potenzen kann gerechnet werden, deswegen finden gewisse Rechengesetzte auch hier eine Anwendung.
\begin{enumerate}
	\item Multiplikation von zwei Variablen mit gleicher Basis, aber mit verschiednen Exponenten. \[a^r\cdot a^s=a^{r+s}\]
	\item Dividieren von zwei Variablen mit gleicher Bais, aber verschiedene Exponenten. \[\frac{a^r}{a^s}=a^{r-s}\]
	\item Basis mit Exponent wird umklammert von einem weiteren Exponenten. \[(a^r)^s=a^{r\cdot s}\]
	\item Die Basis ist ungleich und wird mit einer anderen Baisis multipliziert, aber die Exponenten sind gleich.\[a^k \cdot b^k =(a\cdot b)^k\]
	\item Die Basis ist ungleich wird mit einer anderen Basis dividiert, aber die Exponenten sind gleich. \[\frac{a^k}{b^k}=\left(\frac{a}{b}\right)^k\]
\end{enumerate}
\subsection{Negative Exponenten umformen}\label{sec:Potenzen/Negative Exponenten umformen}
Um einen negativen Exponenten umzuformen muss dieser in ein Bruch geschrieben werden und folgt dabei folgender Form. 

\begin{beispiel}
\begin{align*}
	x^{-4}=\frac{1}{x^4}
\end{align*}
\end{beispiel}
