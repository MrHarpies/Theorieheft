\section{Terme}\label{sec:Terme}
Ein Term ist ein Rechenausdruck, der aus verschiedenen Faktoren und Summanden besteht. Mit einem Term können reelle Sachverhalte oder mathematische Zusammenhänge ausgedrückt werden.

\begin{beispiel}
\begin{enumerate}
	\item Maren kauft zwei Äpfel und drei Bananen
\[\Rightarrow2a+2b\]
\item Zu dem dreifachen einer Unbekannten wird das fünfache einer weiteren Unbekannten addiert
\[\Rightarrow 3x+5y\]
\end{enumerate}
\end{beispiel}

\subsection{Zusammenfassen von Termen}\label{sec:Zusammenfassen von Termen}
Ist ein Term gegeben, so kann man alle Variablen mit dem gleichen Namen zusammenfassen und so den Term vereinfachen. Hierbei werden die Terme der gleichen Sorte zusammengefasst. Weitere Themen zum vereinfachen von Termen sind in den Rechengesetzten zu finden.
%todo Referenz zu Distributiv- Kommutativ- und Assoziativgesetzt

\begin{beispiel}
	\[\color{red}{2a}\color{blue}{-3b} \color{red}{+5a}\color{blue}{-7b}\color{black}{=}\color{red}{7a}\color{black}{+}\color{blue}{-10b} \]
\end{beispiel}
\subsubsection{Vereinfachung mit Faktorisierung}
Nicht nur lassen sich Terme der gleichen ''Sorte'' wie in dem oberen Beispiel zusammenfassen, sondern auch mit der Hilfe der Faktorrisierung. Dies bedeutet, dass man durch Ausklammern der Variablen die Koeffizienten zusammenfassen kann. 

\begin{beispiel}
\begin{align*}
	2a+5a-3b-7b\\
	=a(2+5)+b(-3-7)\\
	=7a+b(-10)
\end{align*}	
\end{beispiel}

