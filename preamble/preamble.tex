

%######################################Packeges#########################################
\usepackage[utf8]{inputenc} % Umlaute im PDF anzeigen
\usepackage{ngerman}
\usepackage{fontenc} 
\usepackage{graphicx} % Anzeigen von Graphiken
\usepackage{xcolor} % Eigene Farben
\usepackage{hyperref} % Für Links
\usepackage{amsmath} % Mathematische Schreibweisen
\usepackage{amssymb} % Mathematische Symbole
\usepackage{tikz}
\usetikzlibrary{shapes.geometric, arrows}
\usepackage{paracol}
\usepackage{gensymb} % Für weitere Zeichen (Grad, usw.)
\usepackage{todonotes} % TODOS
\usepackage{pgfplots}
\pgfplotsset{compat=1.15} % Setzt die Kompatibilität auf die neueste Version
%######################################Farben#########################################

%Farben
\definecolor{darkgreen}{rgb}{0,0.3,0} % Definition von Dunkelgrün

% Tikx Farben
\definecolor{f3551e38-74df-57e2-b793-83d7fe876c85}{RGB}{0, 0, 0}



%######################################Eigene Commands#########################################
     
\newcommand{\link}[2]{

\href{#1}{\textcolor{blue}{\underline{#2}}}}

%######################################GTR-Command#########################################
% Der Command GTR ist über \gtr aufrufbar und erfolgt einen Parameter. Hier muss der name des Buttons in Kleinbuchstaben eingetragen werden. 
% Der Command nutzt das jeweilige Bild aus \Media\GTR\Buttons und formatiert es so, dass es sich in den Text vollkommen einbindet. 

\newcommand{\gtr}[1]{
\hspace{-2.5ex}
    \setlength{\fboxsep}{0pt}% Kein Abstand zwischen Rahmen und Bild
    \setlength{\fboxrule}{0pt}% Dicke des Rahmens
    \raisebox{-0.1\height}{
        \fbox{\includegraphics[height=1.9ex]{Media/GRT/Buttons/#1}}
    }\hspace{-1.5ex}
}

%######################################GTR-Command#########################################

%###################################### Online Materieal Command#########################################

\newcommand{\ExternalLink}{
    \tikz[x=1.2ex, y=1.2ex, baseline=-0.05ex]{% 
        \begin{scope}[x=1ex, y=1ex]
            \clip (-0.1,-0.1) 
                --++ (-0, 1.2) 
                --++ (0.6, 0) 
                --++ (0, -0.6) 
                --++ (0.6, 0) 
                --++ (0, -1);
            \path[draw, 
                line width = 0.5, 
                rounded corners=0.5] 
                (0,0) rectangle (1,1);
        \end{scope}
        \path[draw, line width = 0.5] (0.5, 0.5) 
            -- (1, 1);
        \path[draw, line width = 0.5] (0.6, 1) 
            -- (1, 1) -- (1, 0.6);
        }
    }

\newcommand{\geogebra}[1]{Zu dem behandelten Thema steht eine digitale Visualisierung im Internet bereit.\href{#1}{\ExternalLink}}

\newcommand{\gtrvis}[1]{Der genaue Ablauf ist auf Youtube verfügbar.\href{#1}{\ExternalLink}}



\title{Theorieheft}
\author{Bjarne Axmann}
%######################################Enviroment#########################################

\newenvironment{beispiel}[1][]{
\vspace{0.5cm}
\textbf{\textit{Beispiel.}\em}%
\ifthenelse{\equal{#1}{}}{%
%
}{%

}
}{\hfill$\triangle$}
