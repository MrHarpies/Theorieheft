\usepackage[utf8]{inputenc} % Umlaute im PDF anzeigen
\usepackage{ngerman}
\usepackage{fontenc} 
\usepackage{graphicx} % Anzeigen von Graphiken
\usepackage{xcolor} % Eigene Farben

\usepackage{hyperref} % Für Links
\usepackage{amsmath} % Mathematische Schreibweisen
\usepackage{amssymb} % Mathematische Symbole
\usepackage{tikz}
\usetikzlibrary{shapes.geometric, arrows}
\usepackage{paracol}

\usepackage{movie15} % Für die Einbinung von GIF und Filmen







\usepackage{pgfplots}
\pgfplotsset{compat=newest} % Setzt die Kompatibilität auf die neueste Version




%Farben
\definecolor{darkgreen}{rgb}{0,0.3,0} % Definition von Dunkelgrün

% Tikx Farben

\definecolor{f3551e38-74df-57e2-b793-83d7fe876c85}{RGB}{0, 0, 0}






%Eigene Commands

    % Neue Seite ################################################################################################################################################
    
    
\newcommand{\link}[2]{

\href{#1}{\textcolor{blue}{\underline{#2}}}

}

%######################################GTR-Command#########################################
% Der Command GTR ist über \gtr aufrufbar und erfolgt einen Parameter. Hier muss der name des Buttons in Kleinbuchstaben eingetragen werden. 
% Der Command nutzt das jeweilige Bild aus \Media\GTR\Buttons und formatiert es so, dass es sich in den Text vollkommen einbindet. 

\newcommand{\gtr}[1]{
\hspace{-2.5ex}
    \setlength{\fboxsep}{0pt}% Kein Abstand zwischen Rahmen und Bild
    \setlength{\fboxrule}{0pt}% Dicke des Rahmens
    \raisebox{-0.1\height}{
        \fbox{\includegraphics[height=1.9ex]{Media/GRT/Buttons/#1}}
    }\hspace{-1.5ex}
}

%######################################GTR-Command#########################################


\title{Theorieheft}
\author{Bjarne Axmann}
